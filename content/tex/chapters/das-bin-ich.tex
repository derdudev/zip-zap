\chapter{Das bin ich}

Hier möchte \textbf{ich} schon einmal auf \ref{sec:typographisches-design} verweisen und \eqref{eq:integral-funktion} und direkt die Funktion \(f\) ins Spiel bringen:
\begin{equation}\label{eq:integral-funktion}
    f(x) = \int_0^x g(t)\, dt
\end{equation}
Mit der neuen \texttt{Computer Modern New} Schriftart sehen Indizes noch etwas gewöhnungsbedürftig aus: \(I_n\), \(\{A\}_{m=1}^n\). Ah nein, das war nur bei der \texttt{Computer Modern New Sans} Schriftart so!

\begin{itemize}
    \item {\codefonttwo Computer Modern New}
    \item \texttt{Computer Modern New}
    \item {\codefont Computer Modern New}
\end{itemize}

Mit dem \texttt{unifiedjs} Ecosystem und Funktionen wie \texttt{use(...)} und plugins wie \texttt{unified-latex} oder \texttt{remark}, ist das Parsen von Dateien sehr einfach!

\section{Die Mathematik}\label{sec:die-mathematik}

\section{(Typographisches) Design} \label{sec:typographisches-design}
